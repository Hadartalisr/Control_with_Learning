\documentclass[a4 paper]{article}
% Set target color model to RGB
\usepackage[inner=2.0cm,outer=2.0cm,top=2.5cm,bottom=2.5cm]{geometry}
\usepackage{setspace}
\usepackage[rgb]{xcolor}
\usepackage{verbatim}
\usepackage{subcaption}
\usepackage{amsgen,amsmath,amstext,amsbsy,amsopn,tikz,amssymb}
\usepackage{fancyhdr}
\usepackage[colorlinks=true, urlcolor=blue,  linkcolor=blue, citecolor=blue]{hyperref}
\usepackage[colorinlistoftodos]{todonotes}
\usepackage{rotating}
%\usetikzlibrary{through,backgrounds}
\hypersetup{%
pdfauthor={Ashudeep Singh},%
pdftitle={Homework},%
pdfkeywords={Tikz,latex,bootstrap,uncertaintes},%
pdfcreator={PDFLaTeX},%
pdfproducer={PDFLaTeX},%
}
%\usetikzlibrary{shadows}
% \usepackage[francais]{babel}
\usepackage{booktabs}
\newcommand{\ra}[1]{\renewcommand{\arraystretch}{#1}}

\newtheorem{thm}{Theorem}[section]
\newtheorem{prop}[thm]{Proposition}
\newtheorem{lem}[thm]{Lemma}
\newtheorem{cor}[thm]{Corollary}
\newtheorem{defn}[thm]{Definition}
\newtheorem{rem}[thm]{Remark}
\numberwithin{equation}{section}

\newcommand{\homework}[6]{
   \pagestyle{myheadings}
   \thispagestyle{plain}
   \newpage
   \setcounter{page}{1}
   \noindent
   \begin{center}
   \framebox{
      \vbox{\vspace{2mm}
    \hbox to 6.28in { {\bf 67678 - Introduction to Control with Learning \hfill {\small (#2)}} }
       \vspace{6mm}
       \hbox to 6.28in { {\Large \hfill #1  \hfill} }
       \vspace{6mm}
       \hbox to 6.28in { {\it Instructor: {\rm #3} \hfill Name: {\rm #5} {\rm #6}} }
       %\hbox to 6.28in { {\it TA: #4  \hfill #6}}
      \vspace{2mm}}
   }
   \end{center}
   \markboth{#5 -- #1}{#5 -- #1}
   \vspace*{4mm}
}

\newcommand{\problempoints}[2]{~\\\fbox{\textbf{Problem #1}}\hfill (#2 points)\newline\newline}
\newcommand{\subproblem}[1]{~\newline\textbf{(#1)}}
\newcommand{\D}{\mathcal{D}}
\newcommand{\Hy}{\mathcal{H}}
\newcommand{\VS}{\textrm{VS}}
\newcommand{\solution}{~\newline\textbf{\textit{(Solution)}} }

\newcommand{\important}[1]{\textcolor{blue}{\textit{\textbf{#1}}}} 

\newcommand{\problem}[1]{~\\\fbox{\textbf{Problem #1}} \newline\newline}


\newcommand{\bbF}{\mathbb{F}}
\newcommand{\bbX}{\mathbb{X}}
\newcommand{\bI}{\mathbf{I}}
\newcommand{\bX}{\mathbf{X}}
\newcommand{\bY}{\mathbf{Y}}
\newcommand{\bepsilon}{\boldsymbol{\epsilon}}
\newcommand{\balpha}{\boldsymbol{\alpha}}
\newcommand{\bbeta}{\boldsymbol{\beta}}
\newcommand{\0}{\mathbf{0}}


\begin{document}
\homework{Assignment 1 - LQR Control}{Due: 02/06/24}{Oron Sabag}{}{Hadar Tal}{}
\textbf{Instructions}: 
\begin{itemize}
    \item The assignment is to be done individually.
    \item Submit your assignment as a single PDF file.
    \item This file contains extra materials for those who are interested in learning the tools used in the assignment's solution.
    \item Read all the Questions carefully before you start working on the assignment.
\end{itemize}


\section{Continuous-time System}

In the lecture, we focused on discrete linear systems. However, real-world systems often operate in continuous time, 
making it essential to understand the transition to continuous-time systems and the associated control strategies.

\subsection{Motivation for Continuous-time Systems}
Continuous-time systems are ubiquitous in engineering and natural processes. Examples include electrical circuits, 
mechanical systems, and biological systems. Unlike discrete systems, which are defined at specific time intervals, 
continuous systems evolve over time according to differential equations. This continuous evolution provides a more 
accurate representation of physical phenomena, allowing for precise modeling and control.

\subsection{Equations of the Dynamics}
The state-space representation of a continuous-time linear system is given by the following set of differential equations:

\begin{equation}
    \dot{x}(t) = A x(t) + B u(t),
\end{equation}
\begin{equation}
    y(t) = C x(t) + D u(t),
\end{equation}

where:
\begin{itemize}
    \item $x(t) \in \mathbb{R}^n$ is the state vector.
    \item $u(t) \in \mathbb{R}^m$ is the control input.
    \item $y(t) \in \mathbb{R}^p$ is the output vector.
    \item $A \in \mathbb{R}^{n \times n}$ is the state matrix.
    \item $B \in \mathbb{R}^{n \times m}$ is the input matrix.
    \item $C \in \mathbb{R}^{p \times n}$ is the output matrix.
    \item $D \in \mathbb{R}^{p \times m}$ is the feedthrough (or direct transmission) matrix.
\end{itemize}

\subsection{Linear Quadratic Regulator (LQR) Problem}
The LQR problem for continuous-time systems involves finding a control law that minimizes a quadratic cost function. 
The objective is to regulate the state of the system to the origin while minimizing the control effort. The cost function is defined as:

\begin{equation}
    J = \int_{0}^{\infty} \left( x(t)^T Q x(t) + u(t)^T R u(t) \right) dt
\end{equation}
where $Q \in \mathbb{R}^{n \times n}$ and $R \in \mathbb{R}^{m \times m}$ determine the relative importance of the state and 
control effort in the cost function.


\subsection{Comparison to Discrete-time LQR}
In discrete-time systems, the state-space representation is defined by difference equations rather than differential equations. 
The discrete-time LQR problem is similar to its continuous-time counterpart but with a cost function summed over discrete time steps. 
The key equations are:

\begin{equation}
    x_{k+1} = A_d x_k + B_d u_k, \quad k = 0, 1, 2, \ldots
\end{equation}
\begin{equation}
    J_N(u^N) = \sum_{k=0}^{N} \left( x_k^T Q x_k + u_k^T R u_k \right) + x_{N+1}^T Q_f x_{N+1}
\end{equation}
\begin{equation}
    u_k = -K_d x_k,
\end{equation}
where $K_d$ is the optimal control gain matrix.


\subsection{The Optimal Control}
In the lecture, we discussed the determination of the control gain matrix \( K \) for discrete-time systems using an iterative approach. 
This method involved solving a finite-horizon cost function through backward iteration, essentially using dynamic programming techniques. 
However, for continuous-time systems, the process leverages the \textit{\textbf{Riccati equation}} for an infinite-horizon cost function, 
providing a more direct and analytical solution (\ref{sec:riccati}).

\section{Inverted Pendulum on a Cart}

\subsection{Formulation of the Problem}

\subsection{Systen Dynamics}


\newpage
\textbf{\huge{Extra material}}

\section{Riccati Equation}\label{sec:riccati}


\subsubsection{Lecture Approach}
For discrete-time systems, the control gain matrix \( K \) is determined by solving the discrete-time Algebraic Riccati Equation (ARE). The process involves:

1. Defining the cost function to minimize:
   \[
   J = \sum_{k=0}^{\infty} \left( x_k^T Q x_k + u_k^T R u_k \right),
   \]
   where \( Q \) and \( R \) are weight matrices that penalize the state and control input, respectively.

2. Using backward iteration to solve the Riccati difference equation:
   \[
   P_k = Q + A^T P_{k+1} A - A^T P_{k+1} B (R + B^T P_{k+1} B)^{-1} B^T P_{k+1} A,
   \]
   where \( P_k \) is the cost-to-go matrix.

3. Determining the optimal gain matrix \( K \) from:
   \[
   K = (R + B^T P B)^{-1} B^T P A,
   \]
   which minimizes the cost function and provides the optimal control law \( u_k = -K x_k \).

\subsubsection{Continuous-time Riccati Equation}
For continuous-time systems, the optimal control problem involves minimizing the infinite-horizon quadratic cost function:
   \[
   J = \int_{0}^{\infty} \left( x(t)^T Q x(t) + u(t)^T R u(t) \right) dt.
   \]

The solution involves solving the continuous-time Algebraic Riccati Equation (CARE):
   \[
   A^T P + P A - P B R^{-1} B^T P + Q = 0,
   \]
   where \( P \) is the solution matrix that captures the trade-offs between state regulation and control effort.

The optimal control law is then given by:
   \[
   u(t) = -K x(t),
   \]
   with the gain matrix \( K \) calculated as:
   \[
   K = R^{-1} B^T P.
   \]

The Riccati equation provides a systematic way to find the optimal gain matrix \( K \) for continuous-time systems, ensuring stability and optimal performance. This method leverages the properties of the quadratic cost function and the state-space representation to derive an analytical solution, offering robustness and efficiency in control system design.

\subsection{Comparison to Discrete-time LQR}
In discrete-time systems, the state-space representation is defined by difference equations rather than differential equations. The discrete-time LQR problem is similar to its continuous-time counterpart but with a cost function summed over discrete time steps. The key equations are:

\begin{equation}
x_{k+1} = A_d x_k + B_d u_k,
\end{equation}
\begin{equation}
J = \sum_{k=0}^{\infty} \left( x_k^T Q x_k + u_k^T R u_k \right),
\end{equation}
\begin{equation}
u_k = -K_d x_k,
\end{equation}

where \( K_d \) is obtained from the discrete-time ARE. Despite the differences in their formulations, both continuous and discrete LQR controllers aim to balance state regulation and control effort, providing optimal feedback laws for system stabilization.

By transitioning from discrete to continuous time, we gain a framework that better models real-world dynamics, enabling more precise and effective control strategies.



\newpage
\section{Euler-Lagrange Equations}



\end{document} 
